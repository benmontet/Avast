\documentclass[12pt, letterpaper]{article}

\newcommand{\acronym}[1]{{\small{#1}}}
\newcommand{\project}[1]{\textsl{#1}}
\newcommand{\Avast}{\project{Avast}}

\newcommand{\dd}{\mathrm{d}}

\setlength{\parindent}{1.0\baselineskip}
\linespread{1.09}
\raggedbottom
\sloppy
\sloppypar
\frenchspacing

\begin{document}

\section*{\Avast: A data-driven method for obtaining extremely precise stellar radial velocities}

\noindent
by DWH, others

\paragraph{Abstract:}
In the best radial-velocity experiments, stellar
spectroscopic data are taken with continuously calibrated hardware,
spectra are extracted with very good wavelength solutions, and radial
velocities are determined by cross-correlation with templates.
These approaches have delivered great science---including many
hundreds of planet discoveries---at the meter-per-second level of
precision.
We expect to find far more science---including truly Earth-like
planets around truly Sun-like stars---when these integrated
hardware--software systems deliver centimeter-per-second-ish
precision.
Here we propose a methodological and software component for this new
era of precision: \Avast, a data-driven method for determining
radial velocities from a set of multiple spectra taken with the same
instrument over multiple epochs.
\Avast\ takes a data-driven approach to the radial-velocity
measurement; it delivers only relative velocities, not absolute
velocities, but it does so at nearly the Cram\'er--Rao bound.
Importantly, \Avast\ is capable of locating and modeling changes in
the stellar spectrum that are covariant with velocity.
That is, if some component of a star's apparent velocity shifts are
not due to stellar radial velocity but simply surface convection,
activity, sunspots, or rotation, and these surface effects create
spectral changes that go beyond pure velocity shifts, \Avast\ can find
them and exploit them to improve radial-velocity measurements.
The fundamental technology underlying \Avast\ is regression.

\section{Assumptions and method}

The most important assumptions we make in this work are the following:
\begin{itemize}\itemsep=0ex
\item There are $N$ multiple extracted spectra of the same star, taken
  at (fictitious) Solar-System barycentric times (epochs) $t_n$. Each
  of these spectra has been continuum normalized properly, and is
  represented on a grid of $M$ wavelength pixels $\lambda_m$ in the
  (fictitious) Solar-System barycentric rest frame.
\item At each pixel $m$ of each spectrum $n$ there is Gaussian noise,
  drawn from a distribution with zero mean and known variance
  $\sigma^2_{nm}$.
\item At each epoch $n$, there is a true radial velocity displacement
  $\Delta v_n$ of the (center of mass of the) star, and also some kind
  of spectral distortion.  That is, there is a Doppler shift and a
  spectral change at every epoch.
\item We expect the spectral distortions to be sparse in
  wavelength-space.  That is, we expect only a few, isolated lines to
  participate in any distortion.
\item In the long run (that is, as the number $N$ of epochs grows
  large), there cannot be any true covariance between the spectral
  distortion and the true velocity of the (center of mass of the)
  star. That is, any observed correlation between the spectral
  distortion and the radial velocity must be caused by some kind of
  effect at the stellar surface.
\end{itemize}

Hello World!
\begin{eqnarray}
  y_{nm} &=& \mu(\lambda_m; \Delta v_n) + \mathrm{noise}
  \\
  \mu(\lambda; \Delta v) &=& \mu_0(\lambda)
  + \left.\frac{\dd \mu_0}{\dd\lambda}\right|_{v}\,\frac{\dd\lambda}{\dd v}\,\Delta v
  + \left.\frac{\dd \mu_0}{\dd v}\right|_{\lambda}\,\Delta v
  \\
  \mu_0(\lambda) &=& \sum_{k=0}^{K-1} a_k\,g_k(\lambda)
  \\
  \left.\frac{\dd \mu_0}{\dd\lambda}\right|_{v} &=& \sum_{k=0}^{K-1} a_k\,\frac{\dd g_k}{\dd\lambda}
  \\
  \left.\frac{\dd \mu_0}{\dd v}\right|_{\lambda} &=& \sum_{k=0}^{K-1} b_k\,g_k(\lambda)
  \quad ,
\end{eqnarray}
where [something]

\end{document}
